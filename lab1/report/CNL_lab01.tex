\documentclass[11pt]{article}
\usepackage[margin=1.0in]{geometry}
\addtolength{\topmargin}{0.25in}
\usepackage[document]{ragged2e}


\begin{document}
	{\Huge\textbf{EEE381 Tech Memo}}\\
	\hfill \break
	\textbf{From:} Charles Noah Lutz\\
	\textbf{Partner:} N/A\\
	\textbf{To:} Colin Bussert\\
	\textbf{Date:} Performed: 09/20/18; Due: 09/27/18\\
	\textbf{Subject:} Lab \#01

	\section{Abstract}
	The goal of this exercise was to observe the relationship between the
	voltage and current through a diode and extract the leakage current, \(I_{s}\), and
	the diode non-ideality factor, \(n\). In addition, a half-wave rectifier was simulated
	and built in order to evaluate its performance.

	\section{Theory}
	In order to find the leakage current \(I_{s}\) and the diode non-ideality factor \(n\),
	the diode current and voltage relationship must be understood. The expression for 
	current through a diode can be seen in Equation \ref{equ:i_diode}.

	\begin{equation}
		\label{equ:i_diode}
		i_{D} = I_{s} (\exp(qV_{D}/nkT)-1)
	\end{equation}

	Assuming that \(V_{D} >> 0\), this equation can be simplified slightly to the form
	seen in Equation \ref{equ:i_diode_simple}.
	\begin{equation}
		\label{equ:i_diode_simple}
		i_{D} = I_{s} \exp(qV_{D}/nkT)
	\end{equation}
	
	When the natural log of \(i_{D}\) is plotted against \(V_{D}\), the slope of the
	resulting line is a function of \(n\) (Equation \ref{equ:lni_diode}).
	\begin{equation}
		\label{equ:lni_diode}
		\ln(i_{D}) = \ln(I_{s}) + (\frac{q}{nkT})V_{D}
	\end{equation}

	Assuming room tempurature, the value of \(n\) can be found by looking at the slope
	of the line of Equation \ref{equ:lni_diode}.

	A concise presentation of the theory (1) underlying the analysis of a given
	device or circuit, and/or (2) guiding the design of a circuit to given 
	specifications. Equations must be created using an equation edidtor, not
	cut-and-pasted or hand written. Supporting simulations to validate design
	and analysis, if any were required, should be included here.

	\section{Results and Discussion}
	Tables, graphs, equations, and prose should be used to convey all of the
	results in an easy-to-follow format. Details should be provided to 
	explain how the experimental results were obtained. The text should 
	explain any knowledge and/or information gained by performing the experiment.
	All questions posed in the labratory handout and/or by the TAs in lab should
	be answered.\\
	\hfill \break
	All plots must be created using a software package (e.g. EXCEL or MATLAB).
	Tables and equations must not be hand drawn. Be sure to include comparisions
	between theoretical, simulation, and hardware results, as well as
	comparison to design specifications where appropriate.

	\section{Conclusion}
	A breif section contaning one or two paragraphs that concisely states the
	nature/objective of the assignment, the approch taken to complete the
	assignment, and general observations about the outcome(s). Breif commentary
	on agreement -- or lack thereof -- between theory and experiment would be 
	appropriate, but specific results that were already reported and discussed
	at length in the \textit{Results and Discussion} section should not be 
	repeated.
\end{document}
