\documentclass[11pt]{article}
\usepackage[margin=1.0in]{geometry}
\addtolength{\topmargin}{0.25in}
\usepackage[document]{ragged2e}
\usepackage{graphicx}
\graphicspath{{../pictures/}}
\usepackage{float}
\usepackage{siunitx}


\begin{document}
	{\Huge\textbf{EEE381 Tech Memo}}\\
	\hfill \break
	\textbf{From:} Charles Noah Lutz\\
	\textbf{Partner:} Aaron Smith, Mitchell Darnell\\
	\textbf{To:} Colin Bussert\\
	\textbf{Date:} Performed: 10/04/18; Due: 10/11/18\\
	\textbf{Subject:} Lab \#02

	\section{Abstract}
	The purpose of this lab exercise was to observe the DC and small AC
	signal charactaristics of MOSFET devices. This includes extracting
	various parameters that determine specific characteristics of MOSFETs.
	Theses parameters were extracted from both NMOS and PMOS devices. 
	
	\section{Theory}
	DC behavior of MOSFETs can be modeled like a simple transconductance
	amplifier, where the current through the device is a function of the 
	properties of the transistor, the voltage across the gate and source,
	and the voltage across the drain and source. This relationship changes
	based on what operating mode the transistor is in, either linear or 
	saturation. Both of these relationships can be seen in Equations
	\ref{equ:id_linear} and \ref{equ:id_saturation}.

	\begin{equation}
		\label{equ:id_linear}
		I_d=k_n\prime\frac{W}{L}[(V_{GS}-V_{t})V_{DS} - \frac{1}{2} V_{DS}^2] (1 + \lambda V_{DS}), \qquad V_{DS} \geq V_{GS} - V_{t} \quad(Linear)
	\end{equation}
	\begin{equation}
		\label{equ:id_saturation}
		I_d=k_n\prime \frac{W}{L}(V_{GS}-V_{t})^2 (1+ \lambda V_{DS}), \qquad \qquad \quad V_{DS} < V_{GS}-V_{t} \quad (Saturation)
	\end{equation}
	
	Where \(k_n\prime = \mu_n C_{ox}\), \(\mu_n\) is the electron mobility, \(C_{ox}\)
	is oxide capacitance per unit area, W is channel width, L is channel length,
	\(\lambda\) is the channel length modulation parameter, and \(V_t\) is the 
	threshold voltage. \(V_t\) can be calculated when there is a voltage
	between the source and body with Equation \ref{equ:vt}.

	\begin{equation}
		\label{equ:vt}
		V_t = V_{t0} + \gamma \Big[\sqrt{2\phi_f+V_{SB}} - \sqrt{2\phi_f} \,\Big]
	\end{equation}

	Where \(V_{t0}\) is the threshold voltage when \(V_{SB} = 0\), \(\phi_f\) is a
	physical parameter of the device and \(\gamma\) is the body-effect parameter,
	which can be expressed as seen in Equation \ref{equ:body-effect}.\\

	\begin{equation}
		\label{equ:body-effect}
		\gamma = \frac{\sqrt{2q\varepsilon_sN_{sub}}}{C_{ox}}
	\end{equation}

	The body-effect parameter \(\gamma\) can be found by ploting 
	\(\sqrt{I_D}\) vs. \(V_{GS}\) and finding the slope, which is \(\gamma\).
	Using this same graph, the threshold voltages can be found by looking
	at the x-intercept for each of the different \(V_{SB}\) values.\\

	\hfill \break
	
	These equations apply both to NMOS and PMOS but when calculating values for
	PMOS, the absolute value of the voltages need to be used, due to the opposite 
	polarity of the device.\\

	\hfill\break

	The channel length modulation effect causes an increase in current
	by a factor of \(1 + \lambda V_{DS}\) due to the shortening of the 
	channel length. For small signal AC modeling, this can effectvly be
	modeled by a resistance between the drain and source, \(r_0\). The 
	value can be found using Equation \ref{equ:channel_resistance}.

	\begin{equation}
		\label{equ:channel_resistance}
		r_0 = \frac{1}{|\lambda|I_D} = \frac{|V_A|}{I_D}
	\end{equation}
	
	Where \(V_A\) is the early voltage, which can be found by extrapolating
the line created by the \(I_d\) vs. \(V_{DS}\) curve when the MOSFET is in
	saturation, to the x-intercept and taking the absolute value. 

	\section{Results and Discussion}
	For the first part of the experiment, five different \(V_{DS}\) values were
	applied and \(I_D\) was measured for three \(V_{SB}\) values, with 
	\(V_{GS} = V_{DS}\)for a total of 24 data points. This data was then used 
	to create the \(sqrt{I_D}\) vs. \(V_{DS}\) graphs (Figures \ref{fig:nmos_vt}
	and \ref{fig:pmos_vt} and Tables \ref{table:nmos_vt} and \ref{table:pmos_vt}). 
	
	\begin{figure}[H]
		\centering
		\includegraphics[width=4 in]{nmos_vt.png}
		\caption{NMOS \(\sqrt{I_D}\) vs. \(V_{DS}\)}
		\label{fig:nmos_vt}
	\end{figure}

	\begin{table}[H]
		\centering
		\caption{NMOS \(I_D\) vs. \(V_{DS}\) data}
		\label{table:nmos_vt}
		\begin{tabular}{|r||c|c|c|}
			\hline
			{\(\mathbf{V_{DS}}\) (\si\volt)} & \multicolumn{3}{|c|}{\(\mathbf{I_D}\) (\si{\micro\ampere})}\\
			\hline
			\(V_{SB}\) & 0\si{\volt} & 2\si{\volt} & 5\si{\volt}\\
			\hline
			0 & 0 & 0 & 0\\
			0.5 & 0 & 0 & 0\\
			1 & 0 & 0 & 0\\
			1.5 & 22.2 & 0 & 0\\
			2 & 234.1 & 0 & 0\\
			2.5 & 673.2 & 0 & 0\\
			3 & 1294.4 & 1.2 & 0\\
			3.5 & 2062.8 & 131.3 & 0\\
			4 & 2950.5 & 555.9 & 0\\
			\hline
		\end{tabular}
	\end{table}

	\begin{figure}[H]
		\centering
		\includegraphics[width=4 in]{pmos_vt.png}
		\caption{PMOS \(\sqrt{I_D}\) vs. \(V_{DS}\)}
		\label{fig:pmos_vt}
	\end{figure}

	\begin{table}[H]
		\centering
		\caption{PMOS \(I_D\) vs. \(V_{SD}\) data}
		\label{table:pmos_vt}
		\begin{tabular}{|r||c|c|c|}
			\hline
			{\(\mathbf{V_{SD}}\) (\si\volt)} & \multicolumn{3}{|c|}{\(\mathbf{I_D}\) (\si{\micro\ampere})}\\
			\hline
			\(V_{BS}\) & 0\si{\volt} & 2\si{\volt} & 5\si{\volt}\\
			\hline
			0 & 0 & 0 & 0\\
			0.5 & 0 & 0 & 0\\
			1 & 0 & 0 & 0\\
			1.5 & 9.1 & 0 & 0\\
			2 & 177.9 & 50.6 & 5.5\\
			2.5 & 546.8 & 313.4 & 174.7\\
			3 & 1076 & 756 & 553.6\\
			3.5 & 1741.4 & 1346 & 1097.2\\
			4 & 2525 & 2053 & 1780.1\\
			\hline
		\end{tabular}
	\end{table}
	
	A linear regression was used to find a line of best fit for the extraction
	of the body-effect parameter (\(\gamma\)) and the threshold voltage (\(V_t\)).
	The experimental values for \(V_t\) that were found can be seen in Table
	\ref{table:experimental_vt} and the experimental value for \(\gamma\) can
	be seen in Table \ref{table:experimental_body}.

	\begin{table}[H]
		\centering
		\caption{MOSFET \(V_t\) parameter extraction}
		\label{table:experimental_vt}
		\begin{tabular}{|c|c|c|}
			\hline
			\(\mathbf{|V_{SB}|}\) (\si\volt) & \multicolumn{2}{|c|}{\(\mathbf{V_t}\) (\si\volt)}\\
			\hline
			& NMOS & PMOS\\
			\hline
			0 & 0.644 & 0.669\\
			2 & 1.08 & 0.77\\
			5 & 4.8 & 0.839\\
			\hline
		\end{tabular}
	\end{table}

	\begin{table}[H]
		\centering
		\caption{MOSFET \(\gamma\) parameter extraction}
		\label{table:experimental_body}
		\begin{tabular}{|c|c|}
			\hline
			\textbf{MOSFET} & \(\mathbf{\gamma}\)\\
			\hline
			NMOS & 9.6265\\
			PMOS & 12.2729\\
			\hline
		\end{tabular}
	\end{table}
	Tables, graphs, equations, and prose should be used to convey all of the
	results in an easy-to-follow format. Details should be provided to 
	explain how the experimental results were obtained. The text should 
	explain any knowledge and/or information gained by performing the experiment.
	All questions posed in the labratory handout and/or by the TAs in lab should
	be answered.\\
	\hfill \break
	All plots must be created using a software package (e.g. EXCEL or MATLAB).
	Tables and equations must not be hand drawn. Be sure to include comparisions
	between theoretical, simulation, and hardware results, as well as
	comparison to design specifications where appropriate.

	\section{Conclusion}
	A breif section contaning one or two paragraphs that concisely states the
	nature/objective of the assignment, the approch taken to complete the
	assignment, and general observations about the outcome(s). Breif commentary
	on agreement -- or lack thereof -- between theory and experiment would be 
	appropriate, but specific results that were already reported and discussed
	at length in the \textit{Results and Discussion} section should not be 
	repeated.
\end{document}
